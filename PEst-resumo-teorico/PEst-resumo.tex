\documentclass[a4paper,12pt]{article}
\usepackage{graphicx}
\usepackage[a4paper, total={6in, 9in}]{geometry}
\usepackage[T1]{fontenc}
\usepackage[utf8]{inputenc}
\usepackage{graphicx}
\usepackage{tikz}
\usepackage{float}
\usepackage{mathtools}
\usepackage{fancyhdr}
\usepackage{caption}
\usepackage{textgreek}
\usepackage{yfonts}
\usepackage{amssymb}
\usepackage{hyperref}
\usepackage{amsmath}
\hypersetup{
    colorlinks=true,
    linkcolor=blue,
    filecolor=magenta,
    urlcolor=cyan,
    pdftitle={Overleaf Example},
    pdfpagemode=FullScreen,
    }

\urlstyle{same}
\let\empty\varnothing
\let\eqv\Longleftrightarrow
\usepackage{longtable}
\graphicspath{./images}
\pagestyle{fancy}
\date{Março 2022}
\title{ \\ \large {Initial Report}}
\author{João Barreiros C. Rodrigues}

\begin{document}
        \pagenumbering{gobble}
        \begin{titlepage} % Suppresses displaying the page number on the title page and the subsequent page counts as page 1
        \newcommand{\HRule}{\rule{\linewidth}{0.5mm}} % Defines a new command for horizontal lines, change thickness here
        \center % Centre everything on the page
        \textsc{\LARGE Instituto Superior Técnico}\\[1.5cm] % Main heading such as the name of your university/college
        \textsc{\Large Licenciatura em Engenharia Eletrotécnica e de Computadores}\\[0.25cm]
        \HRule\\[0.4cm]
        {\LARGE\bfseries Probabilidades e Estatística}\\[0.4cm] % Title of your document
        {\huge\bfseries Resumo Teórico}\\[0.4cm] % Title of your document
        \HRule\\[1.5cm]\
        João \textsc{Barreiros C. Rodrigues},nº 99968 , aka \textsc{Ex-Machina},\\
        \vfill\vfill\vfill % Position the date 3/4 down the remaining page
        {\large 2º semestre 2022} % Date, change the \today to a set date if you want to be precise
        \vfill % Push the date up 1/4 of the remaining page
\end{titlepage}
        \pagenumbering{arabic}
        \newpage
        \tableofcontents
        \clearpage
        \section{Definição Axiomática de Probabilidade, segundo Kolmagorov}
                \par
		\subsection{Consequências da definição axiomática}
		\begin{center}
			{\bfseries Propriedade 0}
			\begin{equation}
				0 \geq P(A) \leq 1 , \forall A \in \mathfrak{A}
			\end{equation}
			{\bfseries Propriedade 1}
			\begin{equation}
				P(\overline{A})=1-P(A) \eqv P(\overline{A})+P(A)=1 = P(\Omega) 
			\end{equation}
			{\bfseries Propriedade 2}
                          \begin{equation}
                                  P(A)=P(A) \eqv P(A)-P(A)=0 \eqv P(\empty)=0=P(\overline{\Omega}) 
                          \end{equation}
			{\bfseries Propriedade 3}
                        \begin{equation}
                                  P(A\backslash B) = P(A)-P(A \cap B)
                        \end{equation}
			                             
                        {\bfseries Propriedade 4}
                          \begin{equation}
                                  P(A\cup B)= P(A)+P(B)-P(A\cap B)
                          \end{equation}
		\end{center}
		\subsection{Definição de Probabilidade Condicionada}
			Pode definir-se uma probabilidade condicionada com uma simples proposição mental:
			\begin{center}
				\textit{"Tendo em conta que ocorreu um evento B, qual a probabilidade do evento A suceder."}
			\end{center}
			Assim têm-se, para um evento B com P(B) > 0:
			\begin{equation}
				P(A|B)=\frac{P(A \cap B)}{P(B)}
			\end{equation}
			\subsection{Lei das Probabilidades Compostas}
				\begin{equation}
                                	P(A|B) \times P(B) = P(A \cap B) = P(B|A) \times P(B)
				\end{equation}
	Ou para \textit{n} eventos A$_i$, tal que 0 < P(A$_i$) $\leq$ 1, $\forall$ i, i $\in$ [0, \textit{n}:] 
				\begin{equation}
                			I will do this later
                                \end{equation}


			\subsection{Lei da Probabilidade Total}
				Se A$_i$, $\forall$ i , i $\in$ [1, n] tal que $\forall$ i, A$_i$ $\in$ $\Omega$ $\land$ P(A$_i$)>0 então:
				\begin{equation}
                                          P(B)=\sum{n}{i=1}P(B|A_i)\times P(A_i), \forall B \in \Omega
                                  \end{equation}


				\subsubsection{Teorema de Bayes}
				Se A$_i$ \in \Omega, $\forall$ i , i $\in$ [1, m] formam uma partição de \Omega tal que P(A$_i$) > 0, $\forall$ i, , i $\in$ [1, m]. Então para qualquer evento B com P(B)>0 e qualquer j $\in$ [1, m] tem-se:
				\begin{equation}
					P(A_j | B) =
				\end{equation}
				\begin{equation}
					\frac{P(B|A_j) \times P(A_j)}{P(B)}=
				\end{equation}
				\begin{equation}
					\frac{P(B|A_j) \times P(A_j)}{\sum{m}{i=1} P(B|A_i) \times P(A_i)}
				\end{equation}
	\cleanpage
	\section{Variáveis Aleatórias}
	\subsection{Axiomática Introdutória}
		Dado (r, d, P) uma \bfseries{variável aleatória} (v, a) é uma função:\\
		\begin{equation}
			\begin{split}
				& X: \Omega \longrightarrow \mathbb{R}, \omega \longrightarrow X(\omega) | \\
				& X^{-1} ([-\infty, n]) = { \omega \in \Omega : X(\omega) \leq n } \in \mathcal{A}, \forall n \in \mathbb{R}. \\[0.5cm]
				& Assim: \\
			& P(X=x) : P({\omega \in \Omega : X(\omega )=n })
			\end{split}
		\end{equation}
	\subsection{Função de Distribuição e Tipos de variáveis aleatórias}
		Define-se a função de distribuição (cumulativa) de uma variável aleatória X como:
		\begin{equation}
			\begin{split}
				& F_x: \mathbb{R} \longrightarrow [0,1] | \\
				& F_x(n)=P(X^{-1} ]-\infty, n]) =P(X \leq n)	
			\end{split}
		\end{equation}
		\subsubsection{Propriedades das funções de probabilidade}
			\begin{center}
				\bfseries{Propriedade 0}
				\begin{equation}
					0 \leq Fx(n) \leq 1, \forall n \in \mathbb{R}
				\end{equation}

				\bfseries{Propriedade 1}
				\begin{equation}
					Para n_1 \leq n_2 \Longrightarrow Fx(n_0) \leq Fx(n_1) \therefore F_x \text{ é crescente}
				\end{equation}

				\bfseries{Propriedade 2}
				\begin{equation}
					lim_{n \rightarrow -\infty} F_x(n)=0 
				\end{equation}

				\bfseries{Propriedade 3}
				\begin{equation}
					lim_{n \rightarrow +\infty} F_x(n)=1 
				\end{equation}

				\bfseries{Propriedade 4}
				\begin{equation}
					lim_(n \rightarrow n_0^+) F_x(n) = F_x(n_0^+) \therefore F_x \text{ é contínua à direita}
				\end{equation}

				\bfseries{Propriedade 5}
				\begin{equation}
					P(X=n_0)= F_x(n_0) - lim_(n \rightarrow n_0^+) F_x(n) 
				\end{equation}
	
				\bfseries{Propriedade 6}
				\begin{equation}
					F_x(n_1)-F_x(n_0), P(n_0 < X \leq n_1), forall n_0 < n_1
				\end{equation}
		\end{center}
		\cleanpage
		\subsection{Distribuições tipo de probabilidades discretas}
			\subsubsection{Distribuição de Bernoulli}
			\subsubsection{Distribuição binomial }
			\subsubsection{Distribuição geométrica }
			\subsubsection{Distribuição de Poisson}
		\subsection{Distribuições tipo de probabilidades contínuas}
			\subsubsection{Distribuição de uniforme}
                        \subsubsection{Distribuição exponencial}
                        \subsubsection{Distribuição normal}
		\cleanpage
		\subsection{Pares Aletórios}
			\subsubsection{Definição}
			\subsubsection{Distribuições Marginais}


\end{document}










