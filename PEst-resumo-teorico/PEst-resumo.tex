\documentclass[a4paper,12pt]{article}
\usepackage{graphicx}
\usepackage[a4paper, total={6in, 9in}]{geometry}
\usepackage[T1]{fontenc}
\usepackage[utf8]{inputenc}
\usepackage{graphicx}
\usepackage{tikz}
\usepackage{float}
\usepackage{mathtools}
\usepackage{fancyhdr}
\usepackage{caption}
\usepackage{textgreek}
\usepackage{yfonts}
\usepackage{amssymb}
\usepackage{hyperref}
\hypersetup{
    colorlinks=true,
    linkcolor=blue,
    filecolor=magenta,
    urlcolor=cyan,
    pdftitle={Overleaf Example},
    pdfpagemode=FullScreen,
    }

\urlstyle{same}
\let\empty\varnothing
\let\eqv\Longleftrightarrow
\usepackage{longtable}
\graphicspath{./images}
\pagestyle{fancy}
\date{Março 2022}
\title{ \\ \large {Initial Report}}
\author{João Barreiros C. Rodrigues}

\begin{document}
        \pagenumbering{gobble}
        \begin{titlepage} % Suppresses displaying the page number on the title page and the subsequent page counts as page 1
        \newcommand{\HRule}{\rule{\linewidth}{0.5mm}} % Defines a new command for horizontal lines, change thickness here
        \center % Centre everything on the page
        \textsc{\LARGE Instituto Superior Técnico}\\[1.5cm] % Main heading such as the name of your university/college
        \textsc{\Large Licenciatura em Engenharia Eletrotécnica e de Computadores}\\[0.25cm]
        \HRule\\[0.4cm]
        {\LARGE\bfseries Probabilidades e Estatística}\\[0.4cm] % Title of your document
        {\huge\bfseries Resumo Teórico}\\[0.4cm] % Title of your document
        \HRule\\[1.5cm]\
        João \textsc{Barreiros C. Rodrigues},nº 99968 , aka \textsc{Ex-Machina},\\
        \vfill\vfill\vfill % Position the date 3/4 down the remaining page
        {\large 2$^n$$^d$ semester 2022} % Date, change the \today to a set date if you want to be precise
        \vfill % Push the date up 1/4 of the remaining page
\end{titlepage}
        \pagenumbering{arabic}
        \newpage
        \tableofcontents
        \clearpage
        \section{Definição Axiomática de Probabilidade, segundo Kolmagorov}
                \par
		\subsection{Consequências da definição axiomática}
		\begin{center}
			{\bfseries Propriedade 0}
			\begin{equation}
				0 \geq P(A) \leq 1 , \forall A \in \mathfrak{A}
			\end{equation}
			{\bfseries Propriedade 1}
			\begin{equation}
				P(\overline{A})=1-P(A) \eqv P(\overline{A})+P(A)=1 = \Omega 
			\end{equation}
			{\bfseries Propriedade 2}
                          \begin{equation}
                                  P(A)=P(A) \eqv P(A)-P(A)=0 \eqv P(\empty)=0=\overline{\Omega} 
                          \end{equation}
			{\bfseries Propriedade 3}
                        \begin{equation}
                                  P(A\backslash B) = P(A)-P(A \cap B)
                        \end{equation}
			                             
                        {\bfseries Propriedade 4}
                          \begin{equation}
                                  P(A\cup B)= P(A)+P(B)-P(A\cap B)
                          \end{equation}
		\end{center}
		\subsection{Proposição da Probabilidade Condicionada}
		Deriva da Ideia de que se um evento B ocorreu, qual a probabilidade do evento A suceder.
		Assim têm-se, para um evento B com P(B) > 0:
		\begin{equation}
			P(A|B)=\frac{P(A \cap B)}{P(B)}
		\end{equation}
		\subsection{Lei das Probabilidades Compostas}
		\subsection{Lei da Probabilidade Total}
			\subsubsection{Teorema de Bayes}

			
\end{document} 
